\documentclass{beamer}
\usetheme{Madrid}
\usefonttheme{professionalfonts}
\usepackage[utf8]{inputenc}
\usepackage[spanish]{babel}
\usepackage{hyperref}

\title{pysentimiento}
\subtitle{Toolkit Multilingüe para Minería de Opiniones en Redes Sociales}
\date{\today}

\begin{document}

\begin{frame}
  \titlepage
  \begin{itemize}
    \item Autores: Juan Manuel Pérez, Mariela Rajngewerc, Juan Carlos Giudici, Damián A. Furman, Franco Luque, Laura Alonso Alemany, María Vanina Martínez
  \end{itemize}
\end{frame}

\begin{frame}{1. Introducción y Motivación}
  \begin{itemize}
    \item La extracción de opiniones y estados de ánimo del texto de redes sociales ha crecido sustancialmente.
    \item Desafíos para investigadores: APIs comerciales, sólo inglés, complejidad técnica.
    \pause
    \item \textbf{Problema central:} recursos escasos → APIs de pago o modelos obsoletos o no disponibles en otros idiomas.
    \pause
    \item \textbf{Volumen de datos:} 500 millones de tweets/día (2021) \- imposible analizar manualmente.
    \pause
    \item \textbf{Aplicaciones:} Análisis de comportamiento del consumidor, campañas políticas, evolución de patrones emocionales
  \end{itemize}
\end{frame}

\begin{frame}{2. ¿Qué es pysentimiento?}
  \begin{itemize}
    \item Librería Python open source para minería de opiniones y PNL social.
    \item Objetivo: accesibilidad para investigadores y no expertos, y soporte multilingüe.
    \item Basada en Transformers de HuggingFace → aprovecha modelos preentrenados de última generación.
    \item Idiomas soportados: español, inglés, italiano y portugués.
    \item Disponible en GitHub y HuggingFace Hub como software libre.
  \end{itemize}
\end{frame}

\begin{frame}{3. Contribuciones Principales}
  \begin{itemize}
    \item Toolkit multilingüe y de código abierto para minería de opiniones en Python.
    \item Evaluación exhaustiva de modelos preentrenados de vanguardia para distintas tareas e idiomas.
    \item Evaluación de equidad en análisis de emociones (inglés).
    \item Comparativa con otras herramientas open‑source.
    \item Publicación de modelos óptimos dentro de la librería.
  \end{itemize}
\end{frame}

\begin{frame}{4. Tareas Soportadas}
  \begin{enumerate}
    \item \textbf{Análisis de Sentimiento:} positivo/negativo/neutral.
    \item \textbf{Detección de Emociones:} Ekman (ira, asco, miedo, alegría, tristeza, sorpresa) u otras categorías.
    \item \textbf{Detección de Discurso de Odio:} violencia hacia individuos/grupos protegidos. Clasificación binaria o multietiqueta.
    \item \textbf{Detección de Ironía:} significado opuesto al literal, tarea semántica desafiante.
    \item También incluye POS tagging, NER, detección contextualizada de discurso de odio y análisis de sentimiento dirigido (español rioplatense), aunque no se enfocan en el trabajo principal.
  \end{enumerate}
\end{frame}

\begin{frame}{5. Metodología y Modelos Preentrenados}
  \begin{itemize}
    \item Evaluación de modelos generales y especializados en redes sociales.
    \item \textbf{Preprocesamiento Twitter:}
      \begin{itemize}
        \item Repeticiones limitadas (máx 3).
        \item "jajajajaja" → "jaja"
        \item Handles → token especial \texttt{@USER}.
        \item Hashtags → token especial + separación en palabras.
        \item Emojis → representación textual con token especial \texttt{:emoji:}.
      \end{itemize}
    \item Modelos tipo encoder (BERT, RoBERTa, ELECTRA, BERTweet, RoBERTuito, AlBERTo, BERTimbau, BERTweetBR, BERTabaporu).
    \item Fine‑tuning:
      \begin{itemize}
        \item Optimizador Adam, learning rate triangular.
        \item Búsqueda exhaustiva de hiperparámetros.
        \item 10 experimentos por combinación (task + idioma), métrica Macro F1.
      \end{itemize}
  \end{itemize}
\end{frame}

\begin{frame}{6. Evaluación del Rendimiento}
  \begin{itemize}
    \item Modelos Especializados vs. Generales: 
    \begin{itemize} 
        \item Modelos especializados para redes sociales muestran un rendimiento superior en la mayoría de los idiomas: BERTweet para inglés, BERTweetbr y BERTabaporu para portugués, y RoBERTuito para español.
      \end{itemize}
    \item RoBERTuito: rendimiento robusto en múltiples tareas e idiomas, útil para otros como catalán, gallego, euskera.
    \item Selección: modelo top por tarea e idioma; si el rendimiento es similar, se prefiere el modelo monolingüe/especializado.
  \end{itemize}
\end{frame}

\begin{frame}{7. Evaluación de la Equidad}
  \begin{itemize}
    \item Relevancia de mitigar sesgos sistemáticos en IA.
    \item Dificultad: falta de recursos adecuados; se utilizó el Equity Evaluation Corpus (ECC) (Kiritchenko \& Mohammad, 2018).
    \item Criterio de equidad: \textbf{paridad estadística} (Statistical Parity), cuantificado con la métrica de Impacto Dispar (Disparate Impact, DI). Un modelo se considera justo si DI = 1.
    \item Resultados: DI \> 0.8 → sin impacto adverso evidente.
    \item Modelos BERTweet y RoBERTuito "no muestran un sesgo mayor que los modelos de dominio general."
    \item Precaución: corpus ECC "podría no representar el contexto real", y los modelos de lenguaje grandes han demostrado estar "altamente sesgados".
  \end{itemize}
\end{frame}

\begin{frame}{8. Comparación con Otras Herramientas}
  \begin{itemize}
    \item Se comparó vs VADER, TextBlob, Stanza, TweetNLP y Flair (en sentimient y hate detection).
    \item pysentimiento obtuvo rendimiento superior en la mayoría de conjuntos.
    \item Sentimiento: TweetNLP 2.º en Sentiment140; Flair superó en SST‑2.
    \item Discurso de odio: pysentimiento fue mejor que TweetNLP en todos los idiomas excepto en inglés (TweetNLP > pysentimiento).
    \item " Maldición Multilingüe": el rendimiento subóptimo de TweetNLP en idiomas distintos al inglés "podría atribuirse a la maldición de la multilingüidad". 
    En contraste, pysentimiento logra mejores resultados que un modelo multilingüe general, ya que utiliza modelos monolingües especializados. 
  \end{itemize}
\end{frame}

\begin{frame}{9. Conclusiones y Futuro}
  \begin{itemize}
    \item pysentimiento: toolkit multilingüe con rendimiento de vanguardia en minería de opiniones de redes sociales.
    \item Facilita la investigación en textos de redes sociales.
    \item Incluye procedimientos para el diagnóstico de sesgos antes del despliegue.
    \item Código y modelos disponibles en GitHub y HuggingFace.
    \item Futuro:
      \begin{itemize}
        \item Extensión a más idiomas y tareas.
        \item Utilidades de extracción de información.
        \item Análisis contextual, no solo oraciones sueltas.
      \end{itemize}
  \end{itemize}
\end{frame}

\end{document}
